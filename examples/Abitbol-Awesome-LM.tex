%!TEX TS-program = xelatex
%!TEX encoding = UTF-8 Unicode
% Awesome CV LaTeX Template for Cover Letter
%
% This template has been downloaded from:
% https://github.com/posquit0/Awesome-CV
%
% Authors:
% George Abitbol <george@abitbol.org>
% Lars Richter <mail@ayeks.de>
% 
% Francisé et modifier par ordinatous
% Ludovic Marchal <contact@ordinatous.com>
%
% Template license:
% LICENCE CC BY-SA 4.0 (https://creativecommons.org/licenses/by-sa/4.0/)
%
%-------------------------------------------------------------------------------
% CONFIGURATIONS du document
%-------------------------------------------------------------------------------
% FORMAT du papier
% Format A4 par défaut, utiliser 'letterpaper' pour le format US letter:
\documentclass[11pt, a4paper]{awesome-cv}
% LANGUE
% Francisation du document , format de date, syntaxe et correcteur orthographique:
\usepackage[french]{babel}
% MARGES
% Configurer les marges ici:
\geometry{left=1.4cm, top=.8cm, right=1.4cm, bottom=1.8cm, footskip=.5cm}
% FONTS
% Indiquer l'emplacement des fonts ici:
\fontdir[..fonts/]
% COULEUR
% Couleur pour les mise en avant:
% Awesome Couleurs: awesome-emerald, awesome-skyblue, awesome-red, awesome-pink, awesome-orange
%                 awesome-nephritis, awesome-concrete, awesome-darknight
\colorlet{awesome}{awesome-red}
% Décommenter pour choisir d'autres couleurs:
% \definecolor{awesome}{HTML}{CA63A8}

% Coleurs pour le texte:
% Décommenter pour en choisir une:
% \definecolor{darktext}{HTML}{414141}
% \definecolor{text}{HTML}{333333}
% \definecolor{graytext}{HTML}{5D5D5D}
% \definecolor{lighttext}{HTML}{999999}
% ACTIVER Highlights
% Basculer la valeur à false pour suprimmer les mise en avant:
\setbool{acvSectionColorHighlight}{true}

% If you would like to change the social information separator from a pipe (|) to something else
\renewcommand{\acvHeaderSocialSep}{\quad\textbar\quad}


%-------------------------------------------------------------------------------
%	INFORMATIONS PERSONNELLE
%	Commenter les entrées suivantes si elles ne sont pas nécessaires:
%-------------------------------------------------------------------------------
% Available options: circle|rectangle,edge/noedge,left/right
\photo{../examples/profil_02}
\name{George}{Abitbol}
\position{Cosmonaute le plus Classe du Monde{\enskip\cdotp\enskip}Homme le plus Classe du Monde}
\address{Kennedy Space Center, Cap Canaveral, USA}

\mobile{(+82) 10-9030-1843}
\email{george@abitbol.org}
\homepage{www.abitbol.org}
% RESEAUX SOCIAUX
\github{abitbol}
\linkedin{abitbol}
% \gitlab{gitlab-id}
% \stackoverflow{SO-id}{SO-name}
% \twitter{@twit}
% \skype{skype-id}
% \reddit{reddit-id}
% \medium{madium-id}
% \googlescholar{googlescholar-id}{name-to-display}
%% \firstname and \lastname will be used
% \googlescholar{googlescholar-id}{}
% \extrainfo{extra informations}
% CITATION
\quote{« Attention ! ce flim n'est pas un flim sur le « cyclimse ». Merci de votre compréhension »}


%-------------------------------------------------------------------------------
%	INFO du courrier
%	All of the below lines must be filled out
%-------------------------------------------------------------------------------
% INTERLOCUTEUR
\recipient
  {A l'attention des Ressources Humaines}
  {Société ESN.\\1600 Somewhere\\Another town, CA 94043}
% The date on the letter, default is the date of compilation
\letterdate{Le \today}
% TITRE
\lettertitle{Acte de candidature}
% INTRO
\letteropening{Madame, Monsieur;}
% OUTRO
\letterclosing{Cordialement,}
% Any enclosures with the letter
\letterenclosure[Veillez trouver en pièce jointe]{Curriculum Vitae}


%-------------------------------------------------------------------------------
\begin{document}

% Print the header with above personal informations
% Give optional argument to change alignment(C: center, L: left, R: right)
\makecvheader[L]

% Print the footer with 3 arguments(<left>, <center>, <right>)
% Leave any of these blank if they are not needed
\makecvfooter
  {Le \today}
  {George. Abitbol~~~·~~~Acte de candidature}
  {}

% Print the title with above letter informations
\makelettertitle

%-------------------------------------------------------------------------------
%	CONTENU du courrier
%-------------------------------------------------------------------------------
\begin{cvletter}

\lettersection{A propos du film}
À l'occasion de ses soixante-dix ans en 1993, la Warner délivre à Canal+ l'autorisation exceptionnelle d'utiliser les extraits de son catalogue (environ 3 000 titres). Le but officiel était de permettre de monter un petit film promotionnel, avec néanmoins quelques recommandations : ne pas toucher, entre autres, ni à Clint Eastwood ni à Stanley Kubrick7.

Michel Hazanavicius et Dominique Mézerette, y voyant une occasion unique d'outrepasser les problèmes de copyright et de royalties habituels8, réussissent le tour de force de réaliser un long-métrage complet en s'adjoignant les services des authentiques comédiens de doublage de l'époque des personnages détournés : la voix de Raymond Loyer, l'acteur de doublage attitré de John Wayne, et celle de Roger Rudel, la voix familière et nasillarde de Kirk Douglas et Richard Widmark entre autres8. 

\lettersection{Première diffusion}
La première diffusion a lieu sur la chaîne Canal+ le 31 décembre 1993. Une seconde diffusion a lieu en 2004 sur la chaîne Festival (devenue depuis France 4). Le 11 avril 2009, le film est officiellement projeté sur grand écran au centre Georges-Pompidou lors du festival Hors Pistes en présence des deux auteurs2,3.

En 2015, le film n'est toujours pas sorti en VHS ou en DVD. Il connaît néanmoins une nouvelle jeunesse grâce à une version numérisée qui circule sur Internet, faite en 2008-2009 dans les studios de la rue Cognacq-Jay à Paris, par des techniciens de production passionnés, à partir de la Betacam originale4. 

\lettersection{Pourquoi moi?}
Duis sit amet magna ante, at sodales diam. Aenean consectetur porta risus et sagittis. Ut interdum, enim varius pellentesque tincidunt, magna libero sodales tortor, ut fermentum nunc metus a ante. Vivamus odio leo, tincidunt eu luctus ut, sollicitudin sit amet metus. Nunc sed orci lectus. Ut sodales magna sed velit volutpat sit amet pulvinar diam venenatis.



\end{cvletter}


%-------------------------------------------------------------------------------
% Print the signature and enclosures with above letter informations
\makeletterclosing

\end{document}
